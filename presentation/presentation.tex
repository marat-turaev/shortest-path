%%%%%%%%%%%%%%%%%%%%%%%%%%%%%%%%%%%%%%%%%
% Beamer Presentation
% LaTeX Template
% Version 1.0 (10/11/12)
%
% This template has been downloaded from:
% http://www.LaTeXTemplates.com
%
% License:
% CC BY-NC-SA 3.0 (http://creativecommons.org/licenses/by-nc-sa/3.0/)
%
%%%%%%%%%%%%%%%%%%%%%%%%%%%%%%%%%%%%%%%%%

%----------------------------------------------------------------------------------------
%	PACKAGES AND THEMES
%----------------------------------------------------------------------------------------

% \documentclass{beamer}
\documentclass[10pt, pdf,utf8,russian]{beamer}
\usepackage[T2A]{fontenc}
\usepackage[english,russian]{babel}


\mode<presentation> {

% The Beamer class comes with a number of default slide themes
% which change the colors and layouts of slides. Below this is a list
% of all the themes, uncomment each in turn to see what they look like.

%\usetheme{default}
%\usetheme{AnnArbor}
%\usetheme{Antibes}
%\usetheme{Bergen}
%\usetheme{Berkeley}
%\usetheme{Berlin}
%\usetheme{Boadilla}
%\usetheme{CambridgeUS}
%\usetheme{Copenhagen}
%\usetheme{Darmstadt}
%\usetheme{Dresden}
%\usetheme{Frankfurt}
%\usetheme{Goettingen}
%\usetheme{Hannover}
%\usetheme{Ilmenau}
%\usetheme{JuanLesPins}
%\usetheme{Luebeck}
\usetheme{Madrid}
%\usetheme{Malmoe}
%\usetheme{Marburg}
%\usetheme{Montpellier}
%\usetheme{PaloAlto}
%\usetheme{Pittsburgh}
%\usetheme{Rochester}
%\usetheme{Singapore}
%\usetheme{Szeged}
%\usetheme{Warsaw}

% As well as themes, the Beamer class has a number of color themes
% for any slide theme. Uncomment each of these in turn to see how it
% changes the colors of your current slide theme.

%\usecolortheme{albatross}
%\usecolortheme{beaver}
%\usecolortheme{beetle}
%\usecolortheme{crane}
%\usecolortheme{dolphin}
%\usecolortheme{dove}
%\usecolortheme{fly}
%\usecolortheme{lily}
%\usecolortheme{orchid}
%\usecolortheme{rose}
%\usecolortheme{seagull}
%\usecolortheme{seahorse}
%\usecolortheme{whale}
%\usecolortheme{wolverine}

%\setbeamertemplate{footline} % To remove the footer line in all slides uncomment this line
%\setbeamertemplate{footline}[page number] % To replace the footer line in all slides with a simple slide count uncomment this line

%\setbeamertemplate{navigation symbols}{} % To remove the navigation symbols from the bottom of all slides uncomment this line
}

\usepackage{graphicx} % Allows including images
\usepackage{booktabs} % Allows the use of \toprule, \midrule and \bottomrule in tables
\DeclareUnicodeCharacter{00A0}{ }

%----------------------------------------------------------------------------------------
%	TITLE PAGE
%----------------------------------------------------------------------------------------

\title[Поиск путей на графе дорог]{Поиск путей на графе дорог} % The short title appears at the bottom of every slide, the full title is only on the title page

\author{Марат Тураев}

\institute[СПбАУ] % Your institution as it will appear on the bottom of every slide, may be shorthand to save space
{Руководитель: Алексей Гуревич\\СПбАУ МИТ SE}
\date{19 Декабрь 2013} % Date, can be changed to a custom date

\begin{document}

\begin{frame}
\titlepage % Print the title page as the first slide
\end{frame}

% \begin{frame}
% \frametitle{Overview} % Table of contents slide, comment this block out to remove it
% \tableofcontents % Throughout your presentation, if you choose to use \section{} and \subsection{} commands, these will automatically be printed on this slide as an overview of your presentation
% \end{frame}

%----------------------------------------------------------------------------------------
%	PRESENTATION SLIDES
%----------------------------------------------------------------------------------------

\begin{frame}
\frametitle{Постановка задачи и решения}
Поиск кратчаших путей на графе дорог:
\begin{itemize}
	\item Вершины -- перекрестки, ребра -- дороги
	\item США: 26м вершин и 58м ребер
	\item Ориентированный граф
	\item Длина ребер (длина дорог) положительна
	\item Задача в поиске расстояния между парой вершин
\end{itemize}
Алгоритмы:
\begin{itemize}
	\item Dijkstra
	\item A*
	\item Эвристические алгоритмы
\end{itemize}
\end{frame}

\begin{frame}
\frametitle{Нью-Йорк 264346 вершин, 733846 ребер.}
\begin{center}
	\includegraphics[width=0.7\textwidth]{ny_empty.png}
\end{center} 
\end{frame}

\begin{frame}
\frametitle{Алгоритм Дейкстры}
\begin{center}
	\includegraphics[width=0.7\textwidth]{dijkstra_big.png}
\end{center} 
\end{frame}

\begin{frame}
\frametitle{А*}
\begin{center}
	\includegraphics[width=0.7\textwidth]{a_star_big.png}
\end{center} 
\end{frame}

\begin{frame}
\frametitle{Reaches}
Когда ищем путь из Санкт-Петербурга в Москву, не стоит рассмотривать местные дороги в Твери
\begin{center}
	\includegraphics[width=0.7\textwidth]{reaches_motivation.png}
\end{center} 
\end{frame}

\begin{frame}
\frametitle{Reaches}
\begin{itemize}
	\item 
		Пусть есть $P$ путь из $s$ в $t$:
		\begin{center}
		\includegraphics[width=0.7\textwidth]{reaches_img.png}
		\end{center}
	\item Reach $v$ на этом пути $r(v,P) = min(dist(s,v), dist(v,t))$
	\item Reach $v$ на всем графе $r(v) = max(r(v,P))$, по всем кратчайшим путям $P$
	\item Наблюдение:
		\begin{itemize}
			\item Вершина с большим reach находится на каком-то длинном кратчайшем пути
			\item Вершины на магистралях имеют большой reach
			\item Вершины на местных дорогах маленький reach
		\end{itemize}
\end{itemize}
\end{frame}

\begin{frame}
\frametitle{Как использовать Reach?}
Модифицируем известные алгоритмы
\begin{itemize}
	\item 
		Отбросим вершину $w$ если $r(w)<min(d(s,v)+l(v,w),LB(w,t))$:
		\begin{center}
		\includegraphics[width=0.7\textwidth]{reaches_img2.png}
		\end{center}
	\item 
		$LB(w,t)$ -- нижняя граница на оставшийся путь, например Евклидово расстояние
\end{itemize}
\end{frame}

\begin{frame}
\frametitle{Как посчитать Reach?}
Запустим $|V|$ раз Дейкстру
\begin{itemize}
	\item 
		\begin{center}
		\includegraphics[width=0.6\textwidth]{tree_img.png}
		\end{center}
		$r_s(v) = min(depth_s(v),height_s(v))$
	\item Предпосчет 27 часов на графе Нью-Йорка, на больших графах легко не дождаться
	\item Нужен способ быстрее
\end{itemize}
\end{frame}

\begin{frame}
\frametitle{Как посчитать Reach?}
\begin{itemize}
	\item 
		Будем считать не истинный reach, а верхнюю границу
	\item 
		$r(v) \geq \epsilon \implies \exists P (r(v,P) \geq \epsilon)$
	\item 
		Если $r'(v) < \epsilon$ -- верхняя граница найдена ($r(v) = r'(v)$)
	\item 
		Если $r'(v) \geq \epsilon$ -- про верхнюю границу нельзя сказать ничего ($r(v) \geq r'(v)$)
		\begin{center}
			\includegraphics[width=0.7\textwidth]{eps_img.png}
		\end{center} 
\end{itemize}
Строим частичные деревья минимальных путей\\
Время предподсчета 3ч20м.
\end{frame}

\begin{frame}
\frametitle{Dijkstra + Reach}
\begin{center}
	\includegraphics[width=0.7\textwidth]{dijkstra_reaches_big.png}
\end{center} 
\end{frame}

\begin{frame}
\frametitle{А* + Reach}
\begin{center}
	\includegraphics[width=0.7\textwidth]{a_star_reaches_big.png}
\end{center} 
\end{frame}

\begin{frame}
\frametitle{Результаты}
	\begin{tabular}{|c|c|c|} Алгоритм & Время (секунды) & Просмотрено вершин \\
		\hline
		А*&0.026&43166\\
		А* + Reaches&0.006&6683\\
		Dijkstra&0.099&194748\\
		Dijkstra + Reaches&0.016&24532\\
	\end{tabular}
\end{frame}

\begin{frame} 
\frametitle{Какие навыки получены?}
\begin{itemize}
	\item Разработка требовательных к скорости выполнения программ
		\begin{itemize}
			\item google perftools
		\end{itemize}
	\item Алгоритмы препроцессинга для задачи поиска путей
	\item Методы и библиотеки отрисовки графов
\end{itemize}
\end{frame}

\begin{frame}
\Huge{\centerline{Вопросы?}}
\end{frame}

%----------------------------------------------------------------------------------------

\end{document} 